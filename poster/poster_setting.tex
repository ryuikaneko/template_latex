% Definition of some variables and colors
\setlength{\columnsep}{3cm}% how far from middle line
\setlength{\columnseprule}{2mm}% width of middle line
\setlength{\parindent}{0.0cm}
\setlength{\columnsep}{3cm}
%
\setlength{\headheight}{0cm}
\setlength{\headsep}{0cm}
\setlength{\topmargin}{-2.9cm}
\setlength{\oddsidemargin}{-2.55cm}
\setlength{\paperwidth}{120cm}
\setlength{\paperheight}{91.51cm}
\setlength{\textwidth}{120cm}
\setlength{\textheight}{91.51cm}
\setlength{\tausch}{\paperwidth}
\setlength{\paperwidth}{\paperheight}
\setlength{\paperheight}{\tausch}
\setlength{\tausch}{\textwidth}
\setlength{\textwidth}{\textheight}
\setlength{\textheight}{\tausch}
%
%\newrgbcolor{lightblue}{0. 0. 0.80}
%\newrgbcolor{lightred}{0.80 0. 0.}
%\newrgbcolor{lightgreen}{0. 0.80 0.}
\newrgbcolor{lightblue}{0.8 0.8 1.0}
\newrgbcolor{lightred}{1.0 0.8 0.8}
\newrgbcolor{lightgreen}{0.8 1.0 0.8}
\newrgbcolor{white}{1. 1. 1.}
\newrgbcolor{whiteblue}{.80 .80 1.}
\newrgbcolor{darkblue}{.0 .0 .25}
%\newrgbcolor{darkred}{.5 .0 .1}
\newrgbcolor{darkred}{.8 .0 .1}
\newrgbcolor{darkyellow}{1. 1. .5}
\newrgbcolor{whitegreen}{.75 .90 .75}
\newrgbcolor{greenblack}{.0 .2 .0}
\newrgbcolor{darkgreen}{.0 .5 .0}
\newrgbcolor{whitegray}{.75 .75 .75}

% Background
\newcommand{\background}[3]{
  \newrgbcolor{cgradbegin}{#1}
  \newrgbcolor{cgradend}{#2}
  \psframe[fillstyle=gradient,gradend=cgradend,
  gradbegin=cgradbegin,gradmidpoint=#3](0.,0.)(1.\textwidth,-1.\textheight)
}

% Poster
\newenvironment{poster}{
  \begin{center}
  \begin{minipage}[c]{0.95\textwidth}
}{
  \end{minipage} 
  \end{center}
}

% pcolumn
\newenvironment{pcolumn}[1]{
  \begin{minipage}{#1\textwidth}
  \begin{center}
}{
  \end{center}
  \end{minipage}
}

% pbox
\newrgbcolor{lcolor}{0. 0. 0.80}
\newrgbcolor{gcolor1}{1. 1. 1.}
\newrgbcolor{gcolor2}{.80 .80 1.}
\newcommand{\pbox}[4]{
\psshadowbox[#3]{
\begin{minipage}[t][#2][t]{#1}
#4
\end{minipage}
}}
\newcommand{\ppbox}[4]{
\psframebox[#3]{
\begin{minipage}[t][#2][t]{#1}
#4
\end{minipage}
}}

% mytitle
\newcommand{\mytitle}[1]{
\begin{center}
\ppbox{0.91\columnwidth}{}
%\ppbox{0.85\columnwidth}{}
{linewidth=0mm,framearc=0.0,linecolor=lightblue,
linestyle=none,
fillstyle=gradient,gradangle=90,gradbegin=white,
gradend=whitegray,
%gradend=whitegreen,
gradmidpoint=0.0,framesep=.25em}
{\begin{center}\huge%
\color{darkblue}
%\color{darkgreen}
#1
\end{center}}
\end{center}
\vspace{.5cm}
}

% mysubtitle
\newcommand{\mysubtitle}[1]{
\vspace{0.5cm}
%\begin{center}
{\hspace{30pt}\color{darkblue}
%{\hspace{30pt}\color{darkred}
%{\hspace{90pt}\color{darkred}
\Large\raisebox{.1em}{$\blacksquare$}
#1
}\\
}
%\newcommand{\mysubtitle}[1]{
%\vspace{0.5cm}
%\begin{center}\color{darkred}
%\Large\raisebox{.1em}{$\blacksquare$}
%#1
%\end{center}
%\vspace{0.25cm}
%}

% myfig
%----
% \myfig - replacement for \figure
% necessary, since in multicol-environment 
% \figure won't work
\newcommand{\myfig}[3][0]{
\begin{center}
  \vspace{1.5cm}
  \includegraphics[width=#3\hsize,angle=#1]{#2}
  \nobreak\medskip
\end{center}}

% mycaption
%----
% \mycaption - replacement for \caption
% necessary, since in multicol-environment \figure and
% therefore \caption won't work
\setcounter{figure}{1}
\newcommand{\mycaption}[1]{
  \vspace{0.5cm}
  \begin{quote}
    {{\sffamily\bfseries Figure} \arabic{figure}: #1}
  \end{quote}
  \vspace{1cm}
  \stepcounter{figure}
}

% 斜体
\makeatletter
\newbox\tmp@X
\def\slantbox#1#2{{%
        % 箱#2の大きさと位置は変えずに、箱の中身だけ#1ほど傾ける
        % #1には角度を度単位で指定するかあるいは「sl」「it」と書く
        % 「sl」と書くとarctan(1/6)だけ傾く(ほぼ\slの傾き)
        % 「it」と書くとarctan(1/4)だけ傾く(ほぼ\itの傾き)
        % 箱の大きさは変わらないので、傾いた分中身が箱の前後にはみ出す
        % graphicsパッケージ要  #1に角度を指定する場合はeclarith.styも要
        \setbox\tmp@X\hbox{#2}\edef\tmptok@a{#1}%
        \def\tmptok@b{sl}\ifx\tmptok@a\tmptok@b% theta=atan(1/6)=9.46232deg
                \def\@RotDeg{49.7312}\def\sec@Theta{1.01379}%
                \def\csDif@hTheta{.914112}\def\csSum@hTheta{1.07907}%
        \else \def\tmptok@b{it}\ifx\tmptok@a\tmptok@b% theta=atan(1/4)=14.0362deg
                \def\@RotDeg{52.0181}\def\sec@Theta{1.03078}%
                \def\csDif@hTheta{.870324}\def\csSum@hTheta{1.11469}%
        \else
                \Mul{#1}{.5}\@hTheta \Add{45}\@hTheta\@RotDeg
                \DegRad\@hTheta\@hTheta
                \Cos\@hTheta\c@hTheta \Sin\@hTheta\s@hTheta
                \Add\c@hTheta\s@hTheta\csSum@hTheta
                \Sub\c@hTheta\s@hTheta\csDif@hTheta
                \Mul\csSum@hTheta\csDif@hTheta\sec@Theta
                \Div{1}\sec@Theta\sec@Theta
        \fi\fi
        \mbox{\rotatebox{-\@RotDeg}{\scalebox{\csDif@hTheta}[\csSum@hTheta]{%
                \rotatebox{45}{\scalebox{1}[\sec@Theta]{%
                        \smash{\makebox[0pt][l]{\usebox{\tmp@X}}}%
                }}%
        }}\phantom{\usebox{\tmp@X}}}%
}}
\def\slbox{\slantbox{sl}}\def\itbox{\slantbox{it}}
\makeatother

% itemize
\renewenvironment{itemize}%  
{%
   \begin{list}{\parbox{5pt}{$\bullet$}}% 見出し記号・直後の空白を調節
   {%
      \setlength{\topsep}{0pt}
      \setlength{\itemindent}{0pt}
%      \setlength{\leftmargin}{90pt}% 左のインデント
      \setlength{\leftmargin}{60pt}% 左のインデント
%      \setlength{\rightmargin}{50pt}% 右のインデント
      \setlength{\rightmargin}{0pt}% 右のインデント
      \setlength{\labelsep}{25pt}% 黒丸と説明文の間
%      \setlength{\labelwidth}{50pt}% ラベルの幅
      \setlength{\labelwidth}{0pt}% ラベルの幅
      \setlength{\itemsep}{0pt}% 項目ごとの改行幅
      \setlength{\parsep}{0pt}% 段落での改行幅
      \setlength{\listparindent}{0pt}% 段落での一字下り
   }
}{%
   \end{list}%
}
